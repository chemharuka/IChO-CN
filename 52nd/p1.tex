如今,医学领域的研究旨在降低药物剂量至最低,扩大剂量范围,并改善生活质量,确保患者不受药物副作用的影响。
``控释系统''提供了对这一目标的最佳解答。
控释是一种可以制成膜或基质形式的系统。
药物从诸如水凝胶薄膜或球体之类的系统中受控释放,使药物更有效地进入体内。

扑热息痛是一种具有止痛和解热特性的药物活性物质。它是减轻中度/轻度疼痛和发烧的最常用药物之一。

在此问题中,您将检测对乙酰氨基酚活性物质从水凝胶聚合物系统中的控制释放。

\noindent\textbf{化学品}

tips \textbf{表格}

\noindent\textbf{玻璃仪器与装置}

\begin{itemize}
\tightlist
\item
  1个烧杯,100 mL
\item
  1个量筒,100 mL
\item
  1个吸量管,5 mL
\item
  1个带塞锥形瓶,250 mL
\item
  5个带塞锥形瓶,10 mL
\item
  1个洗耳球
\item
  1把刮刀
\item
  6个带塑料盖的试管
\item
  1个称量盘
\item
  1个培养皿
\item
  1根玻璃棒
\item
  1个洗瓶
\item
  1个电磁加热板搅拌器
\item
  1个秒表
\item
  15个一次性塑料移液器,3 mL
\item
  2张坐标纸
\item
  1个尺子
\item
  1支用于玻璃仪器的马克笔
\item
  1台光度计
\item
  2个紫外线可见石英(或优质塑料,\textless{}200 nm)吸收比色皿
\item
  1台天平
\item
  1个搅拌棒
\item
  1个马克笔
\item
  1个温度计
\item
  1台涡旋混合器
\item
  1个热风枪
\item
  小瓶
\item
  水浴
\item
  注射器
\item
  塑料吸管
\end{itemize}

\textbf{pH = 7.4的缓冲液的制备:}在250 mL烧瓶中放入150 mL蒸馏水。
加入1.6 g NaCl,0.4 g KCl,288 mg
Na\textsubscript{2}HPO\textsubscript{4}和48 mg
NaH\textsubscript{2}PO\textsubscript{4}。 加入蒸馏水直至体积为250
mL,混合以溶解溶液中的所有盐。

\textbf{水凝胶的制备:}

\begin{enumerate}
\tightlist
\item 在小瓶(20 mL)中称取不同量的AA单体(5 mL, 72.9 mmol),作为交联剂的MBAA
  (0.026 g,0.17 mmol),对乙酰氨基酚(已知量),作为引发剂的APS (0.1 g,
  0.44 mmol),添加 去离子水(5.0 mL)。
\item 使用涡旋混合器或磁力搅拌器将混合物溶解在小瓶中。
\item 借助注射器将其装入移液器中(事先用热风枪将纸移液器的底部封闭)。
\item 置于60 °C的热水浴中,使这些混合物在移液管中聚合。
\item 从移液管中取出聚合的凝胶,并将其分成相等的几份。
\end{enumerate}

\noindent\textbf{步骤1:绘制校准曲线}

在本节中,你需要准备表1所示的扑热息痛的标准溶液。随后使用紫外可见分光光度计读取每种溶液的吸光度(\(A\)),随后填写到表1空白处。

\textbf{tips 表1}

\begin{enumerate}
\tightlist
\item
  拿一张坐标纸。
\item
  在\(x\)轴上写下标准对乙酰氨基酚浓度,在\(y\)轴上写下相应的吸光度值。
  确认轴的单位。
\item
  用一条直线通过这些点,然后确定该直线的方程式。
  如果您在获取吸光度和浓度之间的线性关系时遇到任何问题,可以重复实验直到获得一条线性线。
\item
  找到校准方程式。
\end{enumerate}

\noindent\textbf{步骤2:从水凝胶体系中释放扑热息痛}

\begin{enumerate}
\tightlist
\item
  打开电磁加热板搅拌器,并在上方放个250 mL烧杯。 向烧杯中加入100 mL
  PBS溶液,并使用温度计将溶液温度调节至37 °C。
\item
  以250 rpm的速度搅拌溶液。
\item
  使用玻璃棒将提供给您的水凝胶样品放入烧杯中,开始释放。
\item
  经过不同的时间后(0, 10, 20, 30, 40, 50 min),将2
  mL溶液转不同的试管中,用塑料盖密封每个试管,然后向烧杯中补充2 mL
  PBS溶液。
\item
  读取收集的所有溶液在243 nm处的吸光度。 使用PBS溶液作为空白。
\item
  填写表2。
\end{enumerate}

\textbf{tips 表2}

\noindent \textbf{计算与分析}

在本节中,你将检测对扑热息痛从水凝胶体系中的释放行为。使用表2中的吸光度值和校准公式。p

\noindent\textbf{P1.1.} 通过下列方程式计算累计药物释放率

\begin{center}
	累计释放率\(=\frac{v_1Xc_i+v_2\sum(c_{i-1})}{m}\times100\%\)
\end{center}

这个方程中:

\(v_1\):PBS溶液的总体积(100 mL)

\(c_i\):扑热息痛的浓度

\(v_2\):样品的体积(2 mL)

\(m\):扑热息痛在水溶胶中的质量

\textbf{tips 表3}

\noindent\textbf{P1.2.} 使用另一张毫米纸,在\(x\)轴上记录释放时间(\(t=\) 0, 10,
20, 30, 40, 50 min),在\(y\)轴上记录累积的药物释放值。
用直线穿过这些点。

\noindent\textbf{P1.3.}
使用表3的图形计算从一个水凝胶系统中释放20\%对乙酰氨基酚所需的时间(以分钟为单位)。

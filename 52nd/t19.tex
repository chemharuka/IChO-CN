第19题:谐振子与刚体转子模型

双原子分子可以抽象为两个由弹簧连接的质点,其弹性势能是偏离平衡距离的函数。因此,我们可以通过谐振子模型计算其振动频率,该方法得到的频率被称为简谐振动频率。谐振子的能量可以通过下面的公式计算:

\[E_n=h\nu (n+\frac{1}{2})\]

其中,\(\nu\)是简谐振动频率,\(h\)是普朗克常数,\(n\)是非负整数。简谐振动的频率可以通过下面的公式计算:

\[\nu = \frac{1}{2\pi} \sqrt{\frac{k}{\mu}}\]

其中\(k\)是力常数,\(\mu\) 是约化质量。

\[\mu=\frac{m_1m_2}{m_1+m_2}\]

其中\(m_1\)和\(m_2\)是两个原子各自的质量。

对于\textsuperscript{12}C\textsuperscript{16}O分子,其力常数为1902.4
N/m。本题中同位素的原子质量用其质量数近似计算。

\textbf{19.1.}
计算\textsuperscript{12}C\textsuperscript{16}O的简谐振动频率(单位Hz)。

\textbf{19.2.}
将\textsuperscript{12}C\textsuperscript{16}O的简谐振动频率单位换算为cm\textsuperscript{--1}。

\textbf{19.3.}
计算\textsuperscript{12}C\textsuperscript{16}O的零点振动能(ZPVE)(单位kcal/mol)。

\textbf{19.4.}
计算\textsuperscript{13}C\textsuperscript{16}O的简谐振动频率(单位Hz)。

\textbf{19.5.}
计算\textsuperscript{12}C\textsuperscript{17}O的简谐振动频率(单位Hz)。

简谐振动模型也可以适用于多原子分子,其\(n_{\mathrm{freq}}\)个振动频率的能量和可以通过下面的公式计算:

\[E_{n_1n_2\cdots n_{n_{\mathrm{freq}}}} = h\sum^{n_{\mathrm{freq}}}_{i=1} \nu_i
(n_i + \frac{1}{2})\]

其中\(\nu_i\)是简谐振动频率,\(h\)是普朗克常数,\(n\)是非负整数。

对于水分子,其简谐振动频率为1649、3832和 3943
cm\textsuperscript{--1}。试将谐振子模型应用在水分子上:

\textbf{19.6.}
计算零点振动能(ZVPE)(单位为J/mol和cm\textsuperscript{--1})。

\textbf{19.7.} 计算其前5个能级(单位cm\textsuperscript{--1})。

刚体转动模型可以用来描述分子的旋转运动。在该模型中双原子分子的键长始终保持不变,其转动能量可以通过下面的公式计算:

\(E_l = \frac{h^2}{8\pi I^2} l (l+1)\)

其中\(I\)是转动惯量,\(l\)是非负整数。转动惯量的计算公式如下所示:

\(I = \mu R^2\)

其中\(\mu\)是约化质量,\(R\)是双原子分子的键长。

在\textsuperscript{12}C\textsuperscript{16}O分子的微波光谱中,最低能量跃迁的频率为115.270
GHz。

\textbf{19.8.}
计算\textsuperscript{12}C\textsuperscript{16}O分子的键长(单位\AA)。

\textbf{19.9.}
预测\textsuperscript{12}C\textsuperscript{16}O分子下面两个吸收的频率(跃迁选率为\Delta l
= ±1)。

\textbf{19.10.}
计算\textsuperscript{12}C\textsuperscript{17}O分子的最低能量吸收频率。

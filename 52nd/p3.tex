在此任务中,你将用分光光度法测定I\textsubscript{2}和吡啶(pyr)之间的络合反应,并确定络合反应的平衡常数\(K\)。
I\textsubscript{2}和I\textsubscript{2}·pyr配合物可以吸收电磁辐射的可见光区域,但是pyr无法吸收,因为它是无色的。
通过分析吡啶浓度改变但总碘浓度恒定的光谱变化,可以确定络合反应的\(K\)。

\noindent\textbf{注意}:除分光光度法测量外,所有操作均应在通风橱中进行。
完成实验后,应将废液和/或化学药品放入废液容器中。

\noindent\textbf{化学品}


tips \textbf{表格}

\noindent\textbf{玻璃仪器与装置}

\begin{itemize}
\tightlist
\item
  分光光度计
\item
  2个紫外可见石英、玻璃或塑料吸收比色皿
\item
  1个带塞容量瓶,50 mL
\item
  6个带塞容量瓶,25 mL
\item
  1个移液器,1 mL
\item
  1个移液器,10 mL
\item
  1个洗耳球
\end{itemize}

\noindent\textbf{试剂与标准溶液}

0.050 M吡啶的环己烷溶液(每个人50 mL,浓度已知)

0.010 M碘的环己烷溶液(每个人25 mL,浓度已知)

\noindent\textbf{步骤}
\begin{enumerate}
\tightlist
\item
  将以下体积的储备溶液移液到六个标记为F-0,F-1,F-2,F-3,F-4和F-5的25
  mL容量瓶中,用环己烷稀释至刻度,混合均匀。
\item
  在样品池和参比池中使用两个带有溶剂的玻璃吸收池,并扫描350至800
  nm之间的波长以记录基线。
\item
  使用此背景记录样品的所有光谱。
\item
  通过从每个光谱中减去空白的吸光度,测量每个光谱中两个最大值的波长处的吸光度值。
\end{enumerate}

tips

根据络合反应: \[
K=\frac{[\mathrm{I_2\cdot pyr}]}{[\mathrm{I_2}][\mathrm{pyr}]}
\] 

考虑一系列解决方案,其中将吡的增量添加到恒定的I\textsubscript{2}中。
令I\textsubscript{2}(0)为I\textsubscript{2}的总浓度(形式为I\textsubscript{2}和I\textsubscript{2}·pyr),我们可以写成:
\[
\mathrm{[I_2]=[I_{2(0)}]-[I_2\cdot pyr]}
\] 

\(K\)可以用下式表述: \[
\frac{[\mathrm{I_2\cdot pyr}]}{[\mathrm{pyr}]}=K[\mathrm{I_2}]\\
\frac{[\mathrm{I_2\cdot pyr}]}{[\mathrm{pyr}]}=K(\mathrm{[I_{2(0)}]-[I_2\cdot pyr]})
\]基于最后一个方程,以\(\frac{[\mathrm{I_2\cdot pyr}]}{[\mathrm{pyr}]}\)对\(\mathrm{[I_2\cdot pyr]}\)作图,斜率为\(-K\)。

如果我们知道\([\mathrm{I_2\cdot pyr}]\),则可以通过物质平衡得到\([\mathrm{pyr}]\):
\[
\mathrm{pyr_0=[total\ pyr]=[I_2\cdot pyr]+[pyr]}
\] 

为了测量\([\mathrm{I_2\cdot pyr}]\),我们使用吸光度值。
假设\(\mathrm{I_2}\)
和\(\mathrm{[I_2\cdot pyr]}\)在波长\(\lambda\)处都有一定的吸光度,但是pyr在该波长处没有吸光度。
让我们假设我们在1
cm的路径长度上测量吸光度值,以便在使用比尔定律时将路径长度忽略。
每个波长上的吸光度是\(\mathrm{I_2}\)和
\(\mathrm{[I_2\cdot pyr]}\)的吸光度之和: \[
A=\varepsilon_{\mathrm{I_2\cdot pyr}}[\mathrm{I_2\cdot pyr}]+\varepsilon_{\mathrm{I_2}}[\mathrm{I_2}]
\] 由于\(\mathrm{[I_2]=[I_{2(0)}]-[I_2\cdot pyr]}\),因此: \[
A=\varepsilon_{\mathrm{I_2\cdot pyr}}[\mathrm{I_2\cdot pyr}]+\varepsilon_{\mathrm{I_2}}[\mathrm{I_2}]-\varepsilon_{\mathrm{I_2}}[\mathrm{I_2\cdot pyr}]
\]
方程中的\(\varepsilon_{\mathrm{I_2}}[\mathrm{I_{2(0)}}]\)是\(A_0\),即没有加入pyr时的吸光度。因此:
\[
A=[\mathrm{I_2\cdot pyr}](\varepsilon_{\mathrm{I_2\cdot pyr}}-\varepsilon_{\mathrm{I_2}})+A_0,\ [\mathrm{I_2\cdot pyr}]=\frac{\Delta A}{\Delta\varepsilon}
\]
其中\(\Delta\varepsilon=\varepsilon_{\mathrm{I_2\cdot pyr}}-\varepsilon_{\mathrm{I_2}}\),而\(\Delta A=A-A_0\)是每次添加吡啶后观察到的吸光度减去初始吸光度。

用\(\frac{[\mathrm{I_2\cdot pyr}]}{[\mathrm{pyr}]}=K(\mathrm{[I_{2(0)}]-[I_2\cdot pyr]})\)代替上式中的\([\mathrm{I_2\cdot pyr}]\),有:
\[
\frac{\Delta A}{[\mathrm{pyr}]}=K\Delta\varepsilon[\mathrm{I_{2(0)}}]-K\Delta A
\] 

这个方程也被称为斯卡查德方程。

\noindent\textbf{分析与计算}

\noindent\textbf{P3.1.}
画出\(\frac{\Delta A}{[\mathrm{free\ pyr}]}\)对\(\Delta A\)的图(斯卡查德图)。

\noindent\textbf{P3.2.} 从斜率确定该络合反应的\(K\)值。。

\noindent\textbf{P3.3.} 使用截距确定\(\Delta\varepsilon\)值。

\noindent\textbf{P3.4.}
使用I\textsubscript{2}的吸收带确定\(\varepsilon_{\mathrm{I_2}}\)值。

\noindent\textbf{P3.5.} 计算\(\varepsilon_{\mathrm{I_2\cdot pyr}}\)值。

\noindent\textbf{P3.6.} 观察该实验中是否存在等吸光点。

\noindent\textbf{P3.7.} 如果有,请解释为什么观察到等吸光点。

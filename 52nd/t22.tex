反应物同时参与两个或更多个反应的情况,被称作平行反应或竞争反应。乙醇脱水、苯酚的硝化和苯的硝化反应即是平行反应的实例。以下反应是一个一级平行反应的例子:

$$\ce{C <-[$k_2$] A ->[$k_1$] 2B}$$

\noindent\textbf{22.1.}
对于上述关于\textbf{A}的一级平行反应,给出\textbf{B}和\textbf{C}浓度关于\textbf{A}的初始浓度和反应时间$t$的表达式。给出\textbf{B}与\textbf{C}的浓度之比。

\noindent 提示:$\int e^{ax}\ \mathrm d x=\frac{1}{a}e^{ax}+c$

\noindent\textbf{22.2.}
定义$k_1+k_2$为\textbf{A}分解反应的有效速率常数$k_{\mathrm{eff}}$。假设有效速率常数满足Arrhenius方程。写出用$k_1$,$k_2$,$E_{a,1}$和$E_{a,2}$表示的有效活化能$E_{\mathbf A,\mathrm{eff}}$,并估算当$k_1=6.2\ \mathrm{min^{-1}}$,$k_2=3.2\ \mathrm{min^{-1}}$ ,$E_{a,1}=35\ \mathrm{kJ\ mol^{-1}}$和$E_{a,2}=60\ \mathrm{kJ\ mol^{-1}}$时$E_{\mathbf A,\mathrm{eff}}$的值。

\noindent 提示:$\frac{\mathrm d}{\mathrm dx}e^{ax}=ae^{ax}$

计算以有效速率常数确定的反应半衰期$t_{1/2}(\mathrm{eff})$。

\noindent\textbf{22.3.} 如果在278 K下,上述一级平行反应中$k_1$和$k_2$的值分别为6.2 min\textsuperscript{--1}和3.2 min\textsuperscript{--1},计算得到相等摩尔浓度的\textbf{B}和\textbf{C}时的温度。(生成\textbf{B}和\textbf{C}的活化能分别是35 kJ mol\textsuperscript{--1}和60 kJ mol\textsuperscript{--1})

\noindent\textbf{22.4.} 若$k_1>k_2$,示意地画出{[}\textbf{A}{]},
{[}\textbf{B}{]}和{[}\textbf{C}{]}浓度变化曲线。

\noindent\textbf{22.5.} 以下给出一个包含可逆步骤的一级平行-连续反应。

$$\ce{C <-[$k_3$] A <-->[$k_1$][$k_2$] B ->[$k_4$] C}$$

反应的相关数据如下:

$k_1$ = 0.109 min\textsuperscript{--1},
$k_2$ = 0.0752 min\textsuperscript{--1},
$k_3$ = 0.0351 min\textsuperscript{--1},
$k_4$ = 0.0310 min\textsuperscript{--1}。

\begin{longtable}[]{@{}lll@{}}
\toprule
时间 (min) & $\theta_{\mathbf{A},t}$ (min) &
$\theta_{\mathbf{B},t}$ (min)\tabularnewline
\midrule
\endhead
12.9 & 6.89 & 3.79\tabularnewline
\bottomrule
\end{longtable}

$$\theta_{\mathbf{A},t}=\int_0^t\frac{[\mathbf A]}{[\mathbf A]_0}\mathrm dt,\ \theta_{\mathbf{B},t}=\int_0^t\frac{[\mathbf B]}{[\mathbf B]_0}\mathrm dt$$

若{[}\textbf{A}{]}\textsubscript{0} = 5 mol
dm\textsuperscript{--3},计算12.9分钟后{[}\textbf{A}{]},
{[}\textbf{B}{]}和{[}\textbf{C}{]}的浓度。


第20题:星际流浪

在未来,人类可能因为地球上的所有资源被榨干而不得不移居到其他的行星之上。假设在新的星球上,标准大气压为2
bar,标准浓度为1 mol
dm\textsuperscript{−3},所有气体都是理想气体。此时,你被钦定判断下面反应在该行星上的平衡常数:

\(XY_4(g) \rightleftharpoons X(s) + 2 Y_2(g)\)

在298K下,\(\Delta _r S^\ominus = 80 J K^{-1}mol^{-1}\)。

\textbf{20.1.} 根据下面提供的信息,计算该反应在298 K下的标准焓变。

图片

\textbf{20.2.} 计算该反应在298 K下的\Delta \_r G\^{}\ominus。

\textbf{20.3.} 计算该反应在298 K下的K\ominus。

\textbf{20.4.} 假定该反应的\Delta \_r
H\^{}\ominus不随温度变化而变化,计算在50 °C下的K。

\textbf{20.5.} 计算在298 K下,当总压为0.2
bar时XY\textsubscript{4}的解离度。

\textbf{20.6. } 提高下面哪一个选项可以提高产量(不定项选择题)。

压力

反应容器的温度

将来,地球上的气候可能会极不稳定,地表温度可能会骤然升高或降低。假设你正在经历这一段气候剧烈变化的过程,你又被钦定去观测生命之源------水相变过程中的热力学。假定此时温度突然下降到--20
°C。

在1 bar压力下,1摩尔的水变成了--20
°C的过冷水,并且在同样的温度下结冰(在此过程中环境的温度始终是--20
°C)。

根据下面的数据:

在0 °C,1 bar下,冰的熔化焓(\(\Delta m H^\ominus\))为6020 J
mol\textsuperscript{--1}

\(C_{p,m}(H_2O(s)) = 37.7 J mol^{-1} K^{-1}\)

\(C_{p,m}(H_2O(l)) = 75.3 J mol^{-1} K^{-1}\)

在--20 °C摄氏度下过冷水结冰的过程中:

\textbf{20.7.} 计算系统的熵变。

\textbf{20.8.} 计算环境的熵变。

\textbf{20.9.} 计算系统和环境总的熵变。

\(\Delta S = C_p ln \frac{T_{终态}}{T_{始态}}\)

\(\Delta S = - \frac{Q_{相变}}{T}\)


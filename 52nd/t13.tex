简单的\textit{d}区元素氧化物例如Fe\textsubscript{3}O\textsubscript{4}或Co\textsubscript{3}O\textsubscript{4}以及许多有关的混合金属化合物有重要的性质。它们有着与矿物尖晶石MgAl\textsubscript{2}O\textsubscript{4}相关的结构并能给出一个AB\textsubscript{2}O\textsubscript{4}的通式。

化学计量的两种过渡金属(A和B)的硝酸盐通过热反应生成尖晶石型AB\textsubscript{2}O\textsubscript{4}晶状固体,其具有面心立方(fcc)晶胞,晶胞由8个AB\textsubscript{2}O\textsubscript{4}组成。取决于两种阳离子(A和B)的位置,尖晶石结构被分为常式尖晶石和反式尖晶石两类。在常式尖晶石中,A\textsuperscript{2+}离子占据四面体空隙,B\textsuperscript{3+}离子占据八面体空隙。但在反式尖晶石中,结构中2+的离子被一半的3+离子取代。

结晶晶体具有晶胞在3个主轴上重复的三维有序结构。在材料中最小的一组能组成重合图案的原子被称为晶胞。晶胞通过沿着主轴重复构成晶体,完全反映了整个晶体的几何形状和结构。面心立方(\textit{fcc})是一种常见的晶体类型。阴离子(X)占据立方体的顶点和面心(每个顶点和每个面心分别算1/8和1/2,因为顶点和面心分别被8个和2个晶胞共用。阳离子(M)占据占据阴离子的空隙。在\textit{fcc}结构中有8个四面体空隙(角落)和4个四面体空隙(1个在中心3个在棱上,每个棱有1/4个四面体空袭)。因此,晶胞组成为M\textsubscript{4}X\textsubscript{4}经验式为MX。然而,尖晶石结构的晶胞由8个面心立方单元组成。

29.746 g盐A与58.202 g盐B在热反应过程中生成24.724 g纯净产物,AB\textsubscript{2}O\textsubscript{4}。在生成尖晶石型化合物过程中,盐A的金属离子保持了它的氧化态但金属离子B经历了氧化。两种盐都包含了相同数目的水分子,金属离子和硝酸根。对尖晶石型化合物元素分析提供了以下数据:6.538 g金属A和11.786 g金属B。假设最终产物为抗磁性固体。考虑前面提供的信息,回答下列问题。

\noindent\textbf{13.1.} 请给出盐A和B的化学式

\noindent\textbf{13.2.} 画出一种络合物离子结构 i)不含
ii)含一个作双齿配体的硝酸根
并识别复合物中是否有反演中心。反演是一种将每个原子通过该中心到达对面位置的操作。

\noindent\textbf{13.3.}
将金属离子放入晶体结构中的合适位置并指出它是常式还是反式尖晶石。

AB\textsubscript{2}O\textsubscript{4}的X射线衍射数据表明晶胞参数为8.085 \AA,它由8个\textit{fcc}单元组成并相当于立方体的棱长。

\noindent\textbf{13.4.}
绘制AB\textsubscript{2}O\textsubscript{4}的\textit{fcc}单元之一,并将原子放入单元中。

\noindent\textbf{13.5.} AB\textsubscript{2}O\textsubscript{4}的密度是多少?
(提示:1 \AA 等于1.0 x 10\textsuperscript{--10} m)

将该尖晶石型化合物与其它过渡金属(M)反应生成M掺杂的AB\textsubscript{2}O\textsubscript{4}。其中M可以选择占据A或B的位置。副产物为AO(A的一氧化物)。

\noindent\textbf{13.6.}
化合物C中M是Mn\textsuperscript{2+},化合物D中M是Ni\textsuperscript{2+},请给出在化合物C和D中Mn\textsuperscript{2+}和Ni\textsuperscript{2+}的位置。假设在八面体场中Ni\textsuperscript{2+}和B\textsuperscript{3+}的分裂能分别是11500 cm\textsuperscript{--1}和20800cm\textsuperscript{--1},配对能为19500 cm\textsuperscript{--1}。

如果掺杂量很少或在某些情况下,在点阵中掺杂的金属离子的行为类似于自由离子(这意味着M的电子仅能感受到周围原子并局限于M和M自己的原子壳)。假设Mn\textsuperscript{2+}在点阵中表现得像自由离子并拥有自己得定域得电子能级。

\noindent\textbf{13.7.}
画出\emph{d}轨道分裂图并确认Mn\textsuperscript{2+}物种是顺磁性还是抗磁性。

磁化率可通过公式计算。

$$\mu(\mathrm{spin\ only}) = \sqrt{n(n+2)}$$

$n$是未成对电子数。然而,一些其它得电子耦合也会影响磁矩因此需要一个修正项。修正项$\alpha$是关于基态的(对于非畸变基态$\alpha=4$,对于畸变基态$\alpha=2$,基态简并度由电子构型决定,例如全满和半满的轨道组成不形成畸变,部分填充轨道组成造成畸变)(对于Mn\textsuperscript{2+}$\lambda=88$ cm\textsuperscript{--1},对于Ni\textsuperscript{2+}$\lambda=-315$ cm\textsuperscript{--1})以及分裂能(对于Mn\textsuperscript{2+}$\Delta=5000$ cm\textsuperscript{-1},对于Ni\textsuperscript{2+}$\Delta=115000$ cm\textsuperscript{-1}),磁矩为:

$$μ_{\mathrm{eff}} =\mu(\mathrm{spin\ only}) (1 −\alpha\lambda/\Delta)$$

磁化率可以通过实验确定,并且与磁矩相互关联。磁矩(如果我们忽略抗磁贡献)具有以下公式:

$$μ_{\mathrm{eff}}= 2.828\sqrt{X_mT}$$

$T$为开尔文温度,$X_m$是摩尔磁化率。

\noindent\textbf{13.8.}
如果样品C和D的重量分别为25.433和25.471 g,产物在25 °C下的磁化率是多少?

\noindent\textbf{13.9.}
将所有金属离子(A,B,Mn\textsuperscript{2+},Ni\textsuperscript{2+})放入晶体中合适位置并填写下面表格。用\textit{t}\textsubscript{2g}代替\textit{d}\textsubscript{xy},\textit{d}\textsubscript{xz},\textit{d}\textsubscript{yz}以及\textit{e}\textsubscript{g}代替$d_{\mathrm{x^2-y^2}}$,$d_{\mathrm{z^2}}$在八面体(\textit{O}\textsubscript{h}),用\textit{t}\textsubscript{2}和\textit{e}轨道在四面体(\textit{T}\textsubscript{d})情况下。如果有畸变,预测畸变类型并展示\textit{d}轨道分裂。

\begin{longtable}[]{@{}lllll@{}}
	\toprule
	\textbf{金属离子} & \textbf{局部对称性} & \textbf{电子构型} &
	\textbf{简并度} & \textbf{畸变类型}\tabularnewline
	\midrule
	\endhead
	& & & &\tabularnewline\midrule
	& & & &\tabularnewline\midrule
	& & & &\tabularnewline\midrule
	& & & &\tabularnewline\midrule
	& & & &\tabularnewline
	\bottomrule
\end{longtable}
